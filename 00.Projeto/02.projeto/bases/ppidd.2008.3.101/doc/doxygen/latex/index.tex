Parallel Programming Interface for Distributed Data (PPIDD) Reference Manual. The Fortran interface subroutine descriptions can be found in \hyperlink{ppidd__fortran_8c}{ppidd\_\-fortran.c}.

Directory and file structure in PPIDD: \small\begin{alltt}\end{alltt}
\normalsize 


\small\begin{alltt}       .                                  PPIDD root directory
       |-- GNUmakefile                    Makefile to build PPIDD
       |-- README                         README for PPIDD
       |-- VERSION                        VERSION number for PPIDD
       |-- ./doc                          Document directory
       |   `-- Doxyfile                       Document configuration file
       |-- ./lib                          Final location of PPIDD library
       |-- ./src                          Source code directory for PPIDD library  
       |   |-- GNUmakefile                    Makefile for src directory
       |   |-- machines.h                     Head file for machine-related settings
       |   |-- mpi\_helpmutex.c                Mutex source file using helper process
       |   |-- mpi\_nxtval.c                   NXTVAL source file 
       |   |-- mpi\_nxtval.h                   NXTVAL header file
       |   |-- mpi\_utils.c                    MPI utility source file 
       |   |-- mpi\_utils.h                    MPI utility header file
       |   |-- mpiga\_base.c                   Source code for distributed data structure 
       |   |-- mpiga\_base.h                   Head file for distributed data structure
       |   |-- mpimutex-hybrid.c              Mutex source file using distributed processes
       |   |-- mpimutex.h                     Mutex header file using distributed processes
       |   |-- ppidd\_c.c                      C interface source code 
       |   |-- ppidd\_c.h                      C interface header file
       |   |-- \hyperlink{ppidd__doxygen_8h}{ppidd\_doxygen.h}                PPIDD document main page file
       |   |-- ppidd\_dtype.h                  PPIDD data type header file
       |   |-- ppidd\_eaf\_c.c                  C interface source code for EAF
       |   |-- ppidd\_eaf\_c.h                  C interface header file for EAF
       |   |-- ppidd\_eaf\_fortran.c            Fortran interface source code for EAF
       |   |-- ppidd\_eaf\_fortran.h            Fortran interface header file for EAF
       |   |-- \hyperlink{ppidd__fortran_8c}{ppidd\_fortran.c}                Fortran interface source code for PPIDD
       |   |-- ppidd\_fortran.h                Fortran interface header file for PPIDD
       |   |-- ppidd\_sf\_c.c                   C interface source code for SF
       |   |-- ppidd\_sf\_c.h                   C interface header file for SF
       |   |-- ppidd\_sf\_fortran.c             Fortran interface source code for SF
       |   |-- ppidd\_sf\_fortran.h             Fortran interface header file for SF
       |   `-- ppidd\_undefdtype.h             Fortran data type header file for C interface
       `-- ./test                         Test code directory for PPIDD library
           |-- GNUmakefile                    Makefile for test directory
           |-- ppidd\_ctest.c                  C test program
           |-- ppidd\_test.F                   Fortran test program
           |-- sizeofctypes.c                 Code for determining the size of C data types
           |-- sizeoffortypes.F               Code for determining the size of Fortran data types
           |-- ppidd\_test.F                   Fortran test program
           `-- timing\_molpro.c                Utility tool for timing\end{alltt}
\normalsize 


\small\begin{alltt}  \end{alltt}
\normalsize 


Some examples of building the PPIDD library: The following examples are tested on a x86\_\-64//Linux machine on which Intel Fortran and C compilers, MPI2-aware Intel Fortran and C compilers, and Intel MPI library are available. Please be aware the options might be different on other machines. \small\begin{alltt}\end{alltt}
\normalsize 


\small\begin{alltt}     1. Build MPI-2 version of PPIDD:
     make MPICC=mpiicc MPIFC=mpiifort INT64=y FFLAGS='-i8'\end{alltt}
\normalsize 


\small\begin{alltt}     or
     make CC=icc FC=ifort INT64=y FFLAGS='-i8' INCLUDE=/software/intel/mpi/3.1/include64 $\backslash$
     MPILIB='-L/software/intel/mpi/3.1/lib64 -Xlinker -rpath -Xlinker $libdir -Xlinker -rpath -Xlinker /opt/intel/mpi-rt/3.1 -lmpi -lmpiif -lmpigi -lrt -lpthread -ldl -L/usr/lib64 -libverbs -lm'     
     (the front part for MPILIB option comes from `mpiifort -show`, and the rear part '-L/usr/lib64 -libverbs -lm' is used to link with Infiniband network.)\end{alltt}
\normalsize 


\small\begin{alltt}     2. Build Global Arrays version of PPIDD.
     Global Arrays should be installed prior to building PPIDD. As mentioned in Global Arrays documentation(\href{http://www.emsl.pnl.gov/docs/global}{\tt http://www.emsl.pnl.gov/docs/global}), 
     there are three possible ways for building GA: (1) GA with MPI; (2) GA with TCGMSG-MPI; and (3) GA with TCGMSG. 
     PPIDD can be built with either of these interfaces.\end{alltt}
\normalsize 


\small\begin{alltt}     (1) GA with MPI:
     make MPICC=mpiicc MPIFC=mpiifort INT64=y FFLAGS='-i8 -Vaxlib' INCLUDE='../../../ga-4-1-1/include /software/intel/mpi/3.1/include64' MPILIB='-L/usr/lib64 -libverbs -lm'\end{alltt}
\normalsize 


\small\begin{alltt}     or
     make CC=icc FC=ifort INT64=y FFLAGS='-i8 -Vaxlib' INCLUDE='../../../ga-4-1-1/include /software/intel/mpi/3.1/include64' $\backslash$
     MPILIB='-I/software/intel/mpi/3.1/include64 -L/software/intel/mpi/3.1/lib64 -Xlinker -rpath -Xlinker $libdir -Xlinker -rpath -Xlinker $\backslash$
     /opt/intel/mpi-rt/3.1 -lmpi -lmpiif -lmpigi -lrt -lpthread -ldl -L/usr/lib64 -libverbs -lm'\end{alltt}
\normalsize 


\small\begin{alltt}     (2) GA with TCGMSG-MPI:
     make MPICC=mpiicc MPIFC=mpiifort INT64=y FFLAGS='-i8 -Vaxlib' INCLUDE=../../../ga-4-1-1/include MPILIB='-L/usr/lib64 -libverbs -lm'\end{alltt}
\normalsize 


\small\begin{alltt}     or
     make CC=icc FC=ifort INT64=y FFLAGS='-i8 -Vaxlib' INCLUDE=../../../ga-4-1-1/include $\backslash$
     MPILIB='-I/software/intel/mpi/3.1/include64 -L/software/intel/mpi/3.1/lib64 -Xlinker -rpath -Xlinker $libdir -Xlinker -rpath -Xlinker $\backslash$
     /opt/intel/mpi-rt/3.1 -lmpi -lmpiif -lmpigi -lrt -lpthread -ldl -L/usr/lib64 -libverbs -lm'\end{alltt}
\normalsize 


\small\begin{alltt}     (3) GA with TCGMSG:
     make CC=icc FC=ifort INT64=y FFLAGS='-i8 -Vaxlib' INCLUDE=../../../ga-4-1-1/include MPILIB=../../../ga-4-1-1/lib/LINUX64/lib/\end{alltt}
\normalsize 


\small\begin{alltt}  \end{alltt}
\normalsize 
 